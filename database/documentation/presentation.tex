\chapter{Présentation}
\section{Historique}
Au début des années 2000, PHP commençait à être largement utilisé pour créer des applications web. Certains frameworks étaient déjà présents, mais ils présentaient souvent des difficultés pour les appréhender et n'étaient pas forcément adaptés aux besoins de l'époque (performance souvent insuffisante en raison d'un chargement systématique de toutes les classes, fonctionnement exclusivement objet, etc.). De plus, ils ne permettaient que difficilement de remplacer certains composants par d'autres.

Des outils comme Smarty, un moteur de templates qui permet de séparer le code HTML du code PHP commençaient à se faire une place. On trouvait également des bibliothèques assez élaborées comme PHPGACL pour gérer les droits de manière particulièrement pertinente.

La gestion des bases de données n'était pas des plus optimales, et souvent un peu trop conceptuelle.

PrototypePHP a été créé pour assembler divers outils disponibles, selon la con-ception qu'en avait l'auteur à l'époque. Il était loin d'être parfait et a évolué de multiples fois, pour intégrer une approche MVC, puis des contraintes de sécurité, etc. Toutefois, les fondements de départ sont restés quasiment identiques :
\begin{itemize}
\item des actions décrites dans un fichier XML, qui était utilisé pour générer le menu en fonction des droits détenus par l'utilisateur (la description du menu est maintenant réalisée dans un fichier XML dédié) ;
\item une gestion des droits basée sur PHPGACL. Si le produit initial a été abandonné, sa philosophie a été conservée, au moins en partie ;
\item une séparation du code PHP et HTML avec l'utilisation de SMARTY ;
\item un accès aux tables de la base de données réalisé par l'intermédiaire d'une classe dédiée à cet usage, ObjetBDD, qui contient des fonctions très simples à manipuler, comme ecrire(\$data), lire(\$id), supprimer(\$id). La connexion à la base de données était initialement basée sur la bibliothèque ADODB, qui a ensuite été remplacée par PDO quand cette classe a été intégrée à PHP ;
\item un support de l'identification selon cinq modalités : base de données, annuaire LDAP, annuaire LDAP puis base de données, connexion par l'intermédiaire d'un serveur CAS, et récupération de l'identification fournie par un serveur PROXY  ;
\item un souci permanent de la performance, lié au passé de son concepteur\footnote{il a commencé sa carrière à une époque où les ressources informatiques étaient rares, chères, et dont la puissance était limitée}.
\end{itemize}

La première version a été publiée en 2008, dans sourceforge (\href{https://sourceforge.net/projects/prototypephp/}{https: //sourceforge.net/projects/prototypephp/}). Depuis quelques années, elle est disponible dans github (\href{https://github.com/equinton/prototypephp}{https://github.com/equinton/prototypephp}).

Si le principe général d'une conception MVC a prévalu depuis plusieurs années, des améliorations récentes, notamment dans la gestion des vues, a été apportée. À partir de septembre 2016, une meilleure gestion des droits a été implémentée pour travailler de manière transparente avec les groupes issus d'un annuaire d'entreprise de type LDAP. 

\section{Gestion des versions}

Le framework est mis à jour en parallèle aux développements de logiciels bâtis à partir de celui-ci. Le code disponible reflète donc les retranscriptions des modifications apportées au gré des évolutions envisagées par son concepteur.

Le numéro de version est consultable dans le fichier \textit{framework/version}. Les consignes de migration d'une version à l'autre sont décrites dans le dossier \textit{framework/upgrade/}.

\section{plugins utilisés}
Les bibliothèques suivantes sont installées dans le framework :
\begin{itemize}
\item pour le code PHP :
\begin{itemize}
\item ObjetBDD (conçu par le développeur du framework), qui gère l'interface avec la base de données ;
\item SMARTY (\url{http://www.smarty.net}), le moteur de templates ;
\item phpCAS (\url{https://wiki.jasig.org/display/CASC/phpCAS}), pour la connexion par l'intermédiaire d'un serveur CAS.

\end{itemize}
\item pour l'affichage et la conception des pages web, le recours au javascript est omniprésent :
\begin{itemize}
\item JQuery, JQueryUI, et des plugins pour les sélections des dates ;
\item DataTables et ses plugins ;
\item OpenLayers pour l'affichage des cartes ;
\item bootstrap pour la prise en compte de l'affichage sur le mode \textit{responsive} ;
\item etc.

\end{itemize}
\end{itemize}

\subsection{Gestion des mises à jour}
Les bibliothèques PHP sont mises à jour par l'intermédiaire de \textit{composer}\footnote{\href{https://getcomposer.org/}{https://getcomposer.org/}}, et sont stockées dans le dossier \textit{vendor}. Elles sont chargées automatiquement \textit{via} autoload.php.
Les bibliothèques Javascript sont mises à jour par l'intermédiaire de \textit{bower}\footnote{\href{https://bower.io/}{https://bower.io/}}, et sont stockées dans \textit{display/bower\_components}.

Elles sont mises à jour régulièrement, mais il est préférable de vérifier si de nouvelles versions sont disponibles avant de procéder à une mise en production.

Des bibliothèques tierces, non gérées par les outils automatiques, peuvent être installées dans le dossier \textit{plugins} (PHP) ou \textit{display/javascript} (Javascript).

\section{Modèle MVC}

Le framework est basé sur un modèle MVC, qui présente les caractéristiques suivantes :
\begin{itemize}
\item le contrôleur est unique, les actions et les droits associés sont décrits dans un fichier unique ;
\item les vues sont héritées d'une classe non instanciable, avec des classes dédiées à l'usage (html via Smarty, ajax, csv, pdf, binaire) ;
\item le modèle est constitué de deux types d'objets : des classes héritées d'ObjetBDD pour gérer les échanges avec la base de données, et des fichiers de script exécutant les modules (ou actions) demandés.
\end{itemize}

Le framework n'a pas une philosophie \og tout objet \fg{}, comme peuvent l'être d'autres, pour tirer parti de la souplesse de PHP. De nombreuses fonctions permettent de faciliter l'écriture du code et limiter sa taille.

Quelques classes génériques sont utilisées (une classe \textit{Message}, par exemple), et l'application recourt fortement aux variables de session. 

\section{Licences}

Le framework est distribué depuis juillet 2019 sous licence MIT.

Voici la liste des licences des composants utilisés :


\begin{longtable}{|>{\raggedright\arraybackslash}p{2cm}|>{\raggedright\arraybackslash}p{6cm}|>{\raggedright\arraybackslash}p{1.5cm}|>{\raggedright\arraybackslash}p{2cm}|}
\hline
\textbf{Composant - version} & \textbf{Site web et usage} & \textbf{Langage} & \textbf{Licence} \\

\hline
\endhead
\hline\endfoot
SMARTY & \url{http://www.smarty.net} Moteur de templates & PHP & LGPL \\

SMARTY-GETTEXT & Traduction des libellés dans Smarty & PHP & LGPL \\



PHPCAS & \url{https://wiki.jasig.org/display/CASC/phpCAS} Identification via un serveur CAS & PHP & Apache 2.0\\

bootstrap & \url{http://getbootstrap.com} Affichage HTML & CSS et Javascript &  MIT\\

js-cookie-master & \url{https://github.com/js-cookie/js-cookie} Gestion des cookies dans le navigateur & Javascript & MIT \\

Datatables & \url{http://www.datatables.net/} Affichage des tables & Javascript & MIT \\

datetime-moment & \url{https://datatables.net/plug-ins/sorting/datetime-moment} Gestion du tri des dates dans Datatables & Javascript & MIT\\

moment & \url{http://momentjs.com} bibliothèque utilisée par le composant précédent pour le tri des dates & Javascript & MIT\\

Jquery & \url{http://jquery.com/} Fonctions d'encapsulation de Javascript & Javascript & Équivalent BSD \\

JqueryUI & \url{http://jqueryui.com/} Composants graphiques associés à Jquery & Javascript & Équivalent BSD \\

jquery-timepicker-addon & \url{https://github.com/trentrichardson/jQuery-Timepicker-Addon} saisie de la date/heure & Javascript & MIT\\

magnific-popup & \url{http://dimsemenov.com/plugins/magnific-popup/} Affichage des images sous forme de pop-up & Javascript & MIT \\

smartmenus & \url{http://www.smartmenus.org} Affichage des menus dans bootstrap & Javascript & MIT\\

c3js & \url{http://c3js.org/} Création de graphiques & Javascript & MIT \\

openlayers & \url{http://openlayers.org/} Affichage de cartes & Javascript & BSD\\

alpaca & \url{http://alpacajs.org/} Gestion de formulaires JSON & Javascript & Apache 2\\

\hline

\caption{Liste des licences des composants les plus fréquemment utilisés}
\end{longtable} 


\section{Sécurité}
Des contrôles de sécurité sont ajoutés régulièrement, au fur et à mesure de l'évolution des menaces. Le framework est conçu pour répondre intégralement au niveau 1 des recommandations de l'ASVS version 4\footnote{\href{https://github.com/OWASP/ASVS/raw/master/4.0/OWASP Application Security Verification Standard 4.0-en.pdf}{https://github.com/OWASP/ASVS/raw/master/4.0/OWASP Application Security Verification Standard 4.0-en.pdf}} et partiellement au niveau 2 (pas de support de l'identification via des mécanismes externes, comme TOTP ou des tokens physiques, par exemple).